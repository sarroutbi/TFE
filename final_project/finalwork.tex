\documentclass[11pt]{article}
\usepackage{hyperref}
\usepackage[spanish]{babel}
\usepackage[utf8]{inputenc}
\usepackage{pdfpages}
\title{\textbf{Medición de calidad de código en proyectos Open Source en base a métricas}}
\author{Sergio Arroutbi Braojos}
\selectlanguage{spanish}
\date{\today}
\usepackage[bottom=10em]{geometry}
\usepackage{amsmath}
\usepackage{mathtools}
\usepackage{pdflscape}
\usepackage{float}

\begin{document}

\hypersetup
{   
pdfborder={0 0 0}
}
   
\maketitle

\pagebreak

\tableofcontents

\pagebreak

\section{Introducción}

Este documento es una aproximación al mundo de la calidad en el software y, más en concreto, en el código fuente del software libre. La calidad del software es medible, así como también lo es el código fuente que permite construir dicho software. Se define como calidad del software al campo de estudio que describe aquellos atributos de los productos software que son deseables.

Para medir la calidad del código fuente, la utilización de las métricas es fundamental. Se conoce como métrica de software a aquella medida cuantitativa que permite conocer en qué grado un sistema, componente o proceso cumple un atributo determinado ~\cite{ieeeglossary:softwareengineeringterminology}.

Las métricas se han convertido en un ornamento más dentro de las suites de herramientas de gestión del ciclo de vida de desarrollo, proporcionando de un solo vistazo la salud del proyecto a través de paneles de métricas.
El principal problema viene a la hora de conocer qué métricas son más importantes, cuáles lo son menos, cuáles deben tener un seguimiento diario, etc. ~\cite{abistock:usingmetrics}

Dicho lo anterior, cabe destacar que en los proyectos de Software Libre, la propia naturaleza del código fuente, que debe ser accesible de forma pública y usable sin restricciones, hace que la calidad de éste sea aún más importante. El Software Libre es, por su naturaleza, transparencia, y la calidad del código fuente no impacta únicamente en los desarrolladores, sino que es a su vez un factor directamente visible para los potenciales usuarios .
La pretensión de este documento es, en la medida de lo posible, y de forma gradual, intentar llegar a aclarar parte de estos conceptos.

\subsection{Objetivos}
A continuación se definen los objetivos de este documento:
\begin{itemize}
\item{El objetivo de este documento es, en primer lugar, enumerar la existencia de algunos modelos de calidad en el software libre, exponiendo qué carencias muestran a la hora de analizar el código fuente}.
\item{Tras ello, se popondrá un método ligero que permita establecer métricas de análisis de código fuente que permita comparar la calidad de proyectos y justificar qué métricas se definen y en base a qué se evalúan dichas métricas. La principal idea no es llegar a un método eficiente, sino más bien justificar la selección y ponderación de las diversas métricas en base a determinados criterios, de forma que se llegue a la personalización de un modelo de calidad propio para el código fuente. Esto permitirá, además, realizar un repaso por los conceptos más importantes relacionados con el diseño orientado a objetos (acoplamiento, cohesión, encapsulación, modularidad, jerarquía, herencia, etc.)}.
\item{Por otro lado, se realizará el estudio, integración e implementación de las herramientas que permitan aglutinar las métricas definidas, así como ponderar cada una de las mismas, para llegar a una evaluación final. Se evaluarán las herramientas existentes en el estado del arte, se realizarán los pasos que permitan integrarlas entre sí y se implmentará un mecanismo que permita su ejecución y sincronización. Se contemplarán, además, aquellas limitaciones existentes, cómo afectan al modelo de calidad establecido y, en qué forma, se podría modificar el modelo para ajustarlo a las herramientas disponibles y al estado del arte de las mismas}.
\item{Finalmente, se realizará un ejemplo práctico de análisis de proyectos de software libre, de forma que se muestre la aplicación del método a través de las herramientas estudiadas para su comparativa. Así, se expondrá la comparativa entre diversos proyectos, de forma que se evalúen dos proyectos de similares características entre sí. Por otro lado, se establecerá el análisis para dos versiones de un mismo proyecto. Esto permitirá conocer el estado del proyecto desde el punto de vista del estado su calidad en función de la progresión del proyecto en el tiempo}.
\end{itemize}

\subsection{Estructura del documento}
Para acometer los objetivos previamente descritos, este documento sigue la siguiente estructura:

\begin{enumerate}
\item{\underline{Introducción}}. Se establecerá una introducción con un resumen de los contenidos del documento. Así se enumerarán en este apartado la misión del documento, los objetivos que éste pretende así como la estructura del documento para enumerar los mismos.
\item{\underline{Modelo de Calidad}}. Como primera aproximación, en este apartado se introducirán los modelos de calidad ya existentes para proyectos de software libre, como OpenBRR, QSOS o QualOSS. Estos modelos, sin embargo, no realizan un análisis avanzado del estado del código y la calidad del mismo, si bien sí que analizan aspectos relacionados con la calidad del código, al menos, indirectamente, como el número de BUGS.
Hilando con esto, se enumerarán distintas métricas que son importantes a la hora de establecer calidad en el software, como por ejemplo desde el número de métodos por clase, número de atributos por clase, dependencias con otras clases, número de parámetros por método o complejidad ciclomática.
Finalmente, en este apartado se debería describir un modelo de calidad basado en métricas software del estilo de OpenBRR pero enfocado única y exclusivamente al análisis de las métricas, con una ponderación que permita establecer la calidad y una justificación de la misma, en cuanto a qué métricas son consideradas más importantes y porqué.
\item{\underline{Herramientas}}. Este apartado permitirá que se describan aquellas herramientas que permiten obtener una o varias de las métricas anteriores, y cómo se pueden combinar dichas herramientas para llegar a la evaluación de calidad final. Se contemplarán, igualmente, herramientas de desarrollo propio o adaptaciones a herramientas ya existentes. Además, se mostrarán diversas gráficas y/o reportes comparativos de la calidad del código para un determinado proyectos.
Así, se podrá mostrar de forma gráfica una comparativa de aquellas métricas más determinantes del sujeto a analizar.
\item{\underline{Caso práctico: MongoDB vs. rethinkdb vs. arangodb}}. En este apartado se realizará el análisis de dos o más proyectos Open Source cualesquiera, de un tamaño similar en líneas de código, escritos en C++. De esta forma, se popdrán concretar los aspectos vistos anteriormente en una comparativa real.
Un ejemplo posible es la comparativa entre bases de datos NoSQL escritas en C++, como pueden ser MongoDB, rethinkdb y arangodb, aunque hay otras opciones. Se puede realizar de igual forma en este apartado más de una comparativa. Se intentará, en todo caso, buscar proyectos escritos en c++, de naturalezas similares, que permitan realizar una comparativa en función de las distintas métricas anteriores.
\item{\underline{Mejoras y posibles trabajos futuros}}. Finalmente, dentro de esta sección, se recogerán mejoras y posibles trabajos futuros que se podrían realizar para mejorar o extender lo que se haya realizado en este proyecto. Se contemplarán, por tanto, opciones de extender el modelo de calidad con más métricas, o bien incluir aspectos que no se han contemplado, extender el modelo para más lenguajes de programación, etc.
\end{enumerate}

\section{Modelo de Calidad}
En esta sección se pretende llegar a un modelo de calidad, basado en métricas, que permita realizar la comparación entre dos o más proyectos de Software Libre, o el estado de uno proyecto a lo largo del tiempo. Así, se estudiará el arte de los modelos de calidad en proyectos de software libre más utilizados en la actualidad, para ver si se pueden reaprovechar en su totalidad, o, al menos, de forma parcial.

Por otro lado, se establecerán diversas métricas que son consideradas en la medición de la calidad de software, tanto desde el punto de vista de programación funcional, como, sobre todo, en el software diseñado en base a orientación a objetos. Además de enumerar las métricas más importantes, se realizará la categorización y priorización de las mismas. Esto permitirá establecer un modelo de calidad en base a métricas según la categorización establecida.

\subsection{Modelos de calidad en proyectos de Software Libre}

Si bien a veces, aunque cada vez menos, los proyectos de Software Libre se consideran caóticos, que su software no es robusto, que no son utilizados por grandes compañías de software o que son, por naturaleza, proyectos que no disponen de soporte, está más que justificado que estas aseveraciones no son más que mitos en torno a este tipo de software ~\cite{oreilly:tenmythsaboutopensourcesoftware}.

El software libre, por estar más expuesto, posee igual o mejor calidad que el software privativo. Además, existen diversos modelos de calidad que permiten establecer la calidad de un proyecto de software libre, y sobre todo comparar la calidad contra otro proyecto de forma que el usuario disponga de información para decantarse en caso de que se tenga que tomar una decisión sobre el software libre que se va a utilizar.

Obviamente, si bien los usuarios finales particulares no someten el software a un modelo de calidad para decantarse por un software libre u otro, y se basan más en una evaluación menos procedimental y más orientada a experiencia de usuario (basada en facilidad de uso, funcionalidad, etc.), sí que es cada vez más común, en el ámbito empresarial, enfrentar varios proyectos de software libre a un modelo de calidad establecido o bien a un modelo de calidad adaptado a las necesidades de la empresa para tomar la decisión de la adquisición final.

Si bien los objetivos de este documento no son realizar un estudio detallado de los modelos de calidad existentes en el Software Libre, sí que resulta interesante presentar dichos modelos para ver qué ofrecen, qué aspectos consideran importantes y qué tipo de evaluación llevan a cabo para establecer sus criterios de calidad.

Antes de hablar de los modelos existentes, cabe destacar que existen dos tipos de modelos de calidad a la hora de evaluar proyectos de Software Libre:

\begin{itemize}
\item{Modelos de calidad ligeros}. Este tipo de modelos son de una implementación más ligera y sencilla respecto a los pesados. Ejemplos de este tipo de modelo son OpenBRR y QSoS.
\item{Modelos de calidad pesados}. Este tipo de modelos, de mayor complejidad y dificultad en su implementación, son considerados más exactos en cuanto a la extensión de su estudio en torno al software.
\end{itemize}

\subsubsection{Modelos Ligeros de Análisis de Calidad del Software Libre}.

Como ya se expuso anteriormente, este tipo de modelos son de sencilla implementación. Así, establecer la calidad de un proyecto de Software Libre puede llegar a ser tan sencillo como rellenar una Hoja de Cálculo con diversas preguntas y ponderar los resultados resultantes. Éste el es caso de OpenBRR.

\begin{itemize}
\item{OpenBRR} (Open Business Readiness Rating). Pese a que en la actualidad se está migrando a un modelo más evolucionado, conocido como OSSpal~\cite{osspal:osspal}, este modelo de calidad ha sido, sin duda, el más sencillo de implementar de los existentes en la actualidad. Como ya se ha comentado, básicamente es una hoja de cálculo que define una serie de super-atributos (conocidos también en el modelo como categorías), con una serie de subcategorías, que no son más que métricas aplicadas a cada una de las categorías. El usuario puede establecer diferentes pesos tanto a la categoría como a las subcategorías. El modelo establece, finalmente, la puntuación final en función de los pesos asignados por el usuario. En la siguiente figura se muestra una de las múltiples categorías del modelo:

\begin{center}
 \begin{figure}[H]
 \begin{center}
   \includegraphics[width=16cm]{img/openbrr_extract00.png}
   \caption{Categoría OpenBRR}
   \label{fig:openbrrcategory}
 \end{center}
 \end{figure}
\end{center}

Las categorías que este modelo propone son las siguientes:
\begin{enumerate}
\item{\underline{Funcionalidad}}. Grado de cumplimiento de las funcionalidades requeridas.
\item{\underline{Usabilidad}}. Desde el punto de vista de experiencia de usuario, tiempops de instalación y configuración, etc.
\item{\underline{Calidad}}. Orientado a número de releases, parches, bugs críticos abiertos o tiempo de resolución de los mismos.
\item{\underline{Seguridad}}. Número de vulnerabilidades moderadas y/o críticas en determinados períodos de tiempo.
\item{\underline{Rendimiento}}. Disponibilidad de benchmarks, configuración de performance+tuning, etc.
\item{\underline{Escalabilidad}}. De forma que se establezca si el software está bien adaptado para ser escalado.
\item{\underline{Arquitectura}}. Se contemplan, dentro de esta categoría, aspectos como la existencia de plugins de terceros, o si se proporcionan APIs (Application Program Interface)
\item{\underline{Soporte}}. Se evalúan aspectos como la actividad en las listas de correo del proyecto, o la existencia de soporte profesional y su calidad.
\item{\underline{Documentación}}. Existencia de documentación variada, desde manuales de usuario, instalación, administración, despliegue o guías de desarrollo.
\item{\underline{Adopción}}. Esta categoría plantea la existencia de libros en torno al proyecto, o bien la existencia de una base de usuarios real medible.
\item{\underline{Comunidad}}. Aspectos como el número de contribuidores únicos de código en los últimos meses son los que se contemplan en esta categoría del modelo.
\item{\underline{Profesionalismo}}. En esta categoría se analizan aspectos como la existencia de un lídel del proyecto o la dificultad que se presenta a la hora de entrar en el equipo de desarrollo.
\end{enumerate}

Como puede observarse en la descripición de la categoría, si bien este modelo no recoge métricas específicas del código fuente, la parte más interesante que presenta es la evaluación de las métricas que realiza, debido, básicamente, a tres aspectos:
\begin{enumerate}
\item{\textbf{Sencillez}}. Este modelo plantea un modo muy sencillo de evaluación de las métricas basado en puntuación más ponderación de cada una de ellas.
\item{\textbf{Flexibilidad}}. La posibilidad de ponderación de cada una de las categorías y subcategorías permite al usuario establecer qué métricas y conjuntos de métricas son las que tienen más impacto a la hora de evaluar la calidad.
\item{\textbf{Extensibilidad}}. El planteamiento que permite este modelo y la fácil implementación de su evaluación permiten extender la evaluación mediante, simplemente, añadir o eliminar categorías o subcategorías según se establezca que es necesario para el usuario final.
\end{enumerate}

Por tanto, si bien OpenBRR no realizar un análisis de métricas de código fuente, su filosofía y la fácil implementación de métricas que el modelo propone pueden servir perfectamente en la evaluación del modelo de calidad del código fuente.

\item{QSOS~\cite{qsos:qsos} (Qualification and Selection of Opensource Software}. Modelo de calidad ligero cuya especificación se establece en el ``QSOS Manifesto''. Básicamente, este manifiesto establece una serie de puntos que definen el modelo de calidad, resumidos en los siguientes objetivos principales:
\begin{enumerate}
\item{Análisis de necesidades y limitaciones en la adopción del software}.
\item{Evaluación funcional y de requisitos técnicos}.
\item{Formalización de metodología}.
\item{Proporción de un método libre}.
\end{enumerate}
Basado en dichos objetivos principales, QSoS define un proceso iterativo que consiste en cuatro puntos independientes para llevar a cabo el modelo de calidad.

Teniendo en cuenta que el objetivo de este documento no es realizar un análisis detallado de cada uno de los modelos de evaluación de código fuente, puede destacarse, de forma muy resumida, que si bien QSoS es considerado un modelo de evaluación ligero, su implementación es más complicada respecto a OpenBRR. Además, al igual que ocurría con OpenBRR, carece de métricas específicas de calidad de software desde el punto de vista del código fuente. Por tanto, este modelo de calidad no añade ninguna mejora respecto a lo ofrecido por OpenBRR.
\end{itemize}

\subsubsection{Modelos Pesados de Análisis de Calidad del Software Libre}

Bajo este capítulo se expone, básicamente, el modelo QualOSS, modelo considerado como modelo pesado de evaluación de calidad en proyectos de Software Libre.
\begin{itemize}
\item{QualOSS~\cite{qualoss:qualoss} (Quality Of Open Source Software)}. Este modelo, fundado en 2006 y financiado principalmente por la unión europea, persiguió rellenar el hueco detectado en el estado del arte de los modelos de evaluación de calidad en el Software Libre.

Este método tiene como primera intención una automatización semi-completa del análisis de los proyectos y de la obtención de sus métricas, mediante la descarga a través de herramientas como FLOSS-metrics, junto a una serie de scripts que permitan la optimización de esta tarea.

La metodología utilizada por este proyecto consistía básicamente en:
\begin{enumerate}
\item{Pasos Iniciales}. En la primera etapa, se realizaron una serie de entrevistas con compañías y se identificaron las prioridades de las mismas a la hora de incorporar el uso corporativo de algún proyecto de software libre.
\item{Modelo Básico de Calidad}. Este modelo establece una metodología GQM(Goal - Question - Metric) donde cada objetivo se divide en diversas preguntas y diversas métricas resultantes de las preguntas.
\item{Consideraciones de la Comunidad}. Finalmente, se establecen atributos indispensables desde el punto de vista de la comunidad, como el tamaño de la misma, la adecuación, la regeneración, carga de trabajo, etc.
\end{enumerate}
\end{itemize}

Como puede observarse, este método hace una especial incidencia en las métricas. Sin embargo, debido, entre otros factores, a la finalización del proyecto y la no disponibilidad de las métricas en cuestión que se plantearon, unido al hecho de que se trata de una metodología pesada, no se puede considerar QualOSS como un modelo del que se puedan aprovechar ciertas herramientas o planteamientos. Sin embargo, el caracter de automatización que plantea en torno a la obtención de métricas es, sin duda, otro factor importante que también debe ser tenido en cuenta. 

En esta sección se ha realizado un repaso de los modelos de calidad existentes en la evaluación de proyectos de Software Libre. Si bien los modelos de calidad no son directamente aplicables, debido al hecho de que carecen de métricas específicas de calidad en el código fuente, cabe destacar que para la implementación de un modelo de calidad de este tipo algunas de las características de los modelos anteriores son deseables:

\begin{enumerate}
\item{\textbf{Sencillez}}. El modelo elegido debe ser sencillo en su aplicación. De esta forma, se podrá asegurar su utilización en un mayor número de entornos.
\item{\textbf{Flexibilidad}}. La posibilidad de ponderación de cada una de las métricas elegidas, ya sea a través de una propuesta de categorías y subcategorías, en la forma en la que lo hace OpenBRR, y permitir al usuario establecer qué métricas y conjuntos de métricas son las que tienen más impacto a la hora de evaluar la calidad en la aplicación del modelo, es una característica muy recomendable a implementar en el modelo de calidad.
\item{\textbf{Extensibilidad}}. El modelo propuesto debe permitir, de forma sencilla, ser extendido con métricas no contempladas. De esta forma, se favorecerá en el modelo la aplicación de cambios para extender el conjunto de categorías y subcategorías de métricas y su posterior evaluación.
\item{\textbf{Automatización}}. Cabe destacar que, a la hora de implementar el modelo, es recomendable adecuar éste para poder automatizar la obtención de resultados. Si bien el modelo puede quedar planteado a través de una hoja de cálculo, sería deseabe establecer un mini-framework de herramientas, a través de scripting o de la implementación de una suite de llamadas a herramientas, de forma que se pudiera automatizar la evaluación de la calidad cumpliendo las características anteriores.
\end{enumerate}

Una vez establecidas las características básicas del modelo, es deseable realizar un estudio de las principales métricas existentes en el análisis del código fuente, tanto desde el punto de vista del diseño orientado a objetos como desde el punto de vista de análisis de código funcional, para establecer en qué medida estas métricas afectan a la calidad y, de esta forma, evaluar el posible impacto de las métricas en el modelo elegido.

\subsection{Métricas de diseño orientado a objetos}

En el apartado de métricas de software orientado a objetos, cabe destacar que hay diversos aspectos a estudiar a la hora de establecer la calidad. A lo largo del tiempo, estas métricas se están usando de forma incremental tanto para evaluar como para predecir la calidad ~\cite{oometrics:introduction}. Un gran número de resultados empíricos soportan la valided de estas métricas ~\cite{validation:oodesignasqualityindicators}, y de forma habitual se usan como un indicador temprano de atributos a los que está relacionada la métrica, como puedan ser la robustez, la mantenibilidad o la propensión a fallos, ya que normalmente, éstos últimos, no pueden ser evaluados hasta una etapa muy tardía dentro del proceso de desarrollo software.

Las métricas descritas en este punto están tomadas del conjunto de métricas MOOD ~\cite{mood:metricsset}. A continuación se exponen las diversas métricas a considerar:

\subsubsection{Acoplamiento}
En 1974, Stevens et. al definieron el acoplamiento como ``la medida de fuerza de asociación establecida por la conexión de un módulo a otro''~\cite{structuredesign:coupling}. Básicamente, el acoplamiento define la interdependencia entre dos objetos. Los objetos A y B están acoplados si un método del objeto A llama a un método del objeto B. Normalmente, un acoplamiento excesivo indica una encapsulación débil y puede afectar a la reutilización del módulo.

\subsubsection{Cohesión}
El concepto de cohesión se refiere a cuan relacionadas están los métodos y atributos de una clase con respecto al resto, esto es, el grado en que los métodos están relacionados con las variables de una clase. El LOCOM (Lack of Cohesion) es una medida que establece el conjunto de métodos locales disjuntos entre sí.

La Cohesión puede medirse de varias formas. Un método para poder medir la cohesión es calcular qué porcentaje de métdos utiliza cada una de las variables y hacer una media de dichos porcentajes.

Una alta cohesión implica una buena subdivisión de clases, mientras que una baja cohesión incrementa la complejidad, aumentando la probabilidad de errores, y es, sin duda, indicativo de un mal diseño y de futuros problemas en la reusabilidad. Las clases poco cohesivas deben ser partidas en clases con más cohesión.

\subsubsection{Encapsulación}
La ocultación de información es un mecanismo de diseño que permite que únicamente un subconjunto de sus métodos son conocidos por los usuarios del módulo. La ocultación de información da lugar a la encapsulación en los lenguajes orientados a objetos. Se considera que los programas con un buen nivel de encapsulación son más fáciles de modificar (hasta en un factor 4) respecto a los programas que no lo son \cite{improving:softwareproductivity}.

Básicamente, la encapsulación se proporciona a través de dos métricas:

\begin{itemize}
\item{AHF (Attribute Hiding Factor)}. Referencia a la invisibilidad de los atributos de una clase para el resto de clases. Lo ideal es que este valor sea de un 100\% (las clases externas no pueden acceder a ningún atributo de la clase).
\item{MHF (Method Hiding Factor)}. Referencia a la invisibilidad de los métodos de una clase para el resto de clases. La ocultación de métodos incrementa la reusabilidad de otras aplicaciones y reduce la complejidad.
\end{itemize}

\subsubsection{Herencia}
La herencia de objetos decrementa la complejidad a través de la reducción del número de operaciones, pero es un tipo de relación entre clases que puede complicar el diseño y el mantenimiento. Se utilizan básicamente dos métricas en torno a la herencia:

\begin{itemize}
\item{DIT (Depth of Inheritance Tree)}. Se define como la profundidad máxima de una clase nodo (que no tiene más clases hijas) dentro del árbol jerárquico de herencia. Cuanto más profunda una clase dentro de una herencia de clases, mayor número de métodos y atributos hereda, haciendo más complejo su diseño, su mantenimiento y su propensión a errores. 
\item{NOC (Number of Children)}. Esta métrica define directamente el número de subclases de una clase. Las clases que tienen un gran número de hijos se consideran difícil a la hora de modificar y se consideran más complejas y propensas a errores.
\end{itemize}

\subsubsection{Complejidad de clase}

El concepto de complejidad de clase viene directamente definido por la métrica WMC (Weighted Methods/Class). Este valor directamente relaciona la cantidad de métodos que tiene una clase. A mayor número de métodos, mayor complejidad (siempre que se tenga un valor razonable que no haga que las clases sean clases perezosas).

\subsubsection{Abstracción}
\subsubsection{Estabilidad}

\subsection{Métricas de calidad de código de clases}
\subsubsection{Número de métodos por clase}
\subsubsection{Número de atributos por clase}

\subsection{Métricas de calidad de código funcional}
\subsubsection{Código duplicado}
\subsubsection{Número de parámetros por método}
\subsubsection{Longitud de métodos}
\subsubsection{Complejidad ciclomática}
\subsubsection{Formateo de código}
\subsubsection{Comentarios de código}

\section{Propuesta de Modelo de Calidad}

\section{Herramientas}

\subsection{cccc}

\subsection{cppdepend}

\subsection{cpd}

\subsection{Sonar}

\section{Un ejemplo práctico: MongoDB vs. rethinkdb vs. arangodb}

\subsection{Análisis de la calidad del código}

Esta sección permitirá concretar el modelo de calidad elegido sobre los proyectos inspeccionados. De esta forma, se realizará una comparativa entre los diversos proyectos a analizar.

\subsection{Histórico de calidad del código}

A través de las herramientas descritas anteriormente, se podrá realizar un análisis de la evolución histórica que ha ido sufriendo el código a lo largo del tiempo. De esta forma se podrá identificar si la coumnidad está 

\section{Mejoras y posibles trabajos futuros}

\pagebreak

\bibliographystyle{alpha}
\bibliography{bibliography}
\label{Bibliography}

\end{document}
