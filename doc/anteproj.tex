\documentclass[11pt]{article}
%Gummi|061|=)
\usepackage{hyperref}
\usepackage[spanish]{babel}
\usepackage[utf8]{inputenc}
\usepackage{pdfpages}
\title{\textbf{Anteproyecto Fin de Experturía: Medición de calidad de código en proyectos Open Source en
base a métricas}}
\author{Sergio Arroutbi Braojos}
\selectlanguage{spanish}
\date{\today}
\usepackage[bottom=14em]{geometry}
\usepackage{amsmath}
\usepackage{mathtools}
\usepackage{pdflscape}

\begin{document}

\hypersetup
{   
pdfborder={0 0 0}
}
   
\maketitle

\pagebreak

\tableofcontents

\pagebreak

\section{Introducción}

Breve introducción que enumere el objetivo que se persigue con este proyecto.

\section{Un modelo de calidad basado en métricas software}

En este apartado se introducirán los modelos de calidad ya existentes para proyectos de software libre, como OpenBRR, QSOS o QualOSS. Hilando con esto, se enumerarán distintas métricas que son importantes a la hora de establecer calidad en el software, como por ejemplo desde el número de métodos por clase, número de atributos por clase, dependencias con otras clases, número de parámetros por método o complejidad ciclomática.

Finalmente, en este apartado se debería describir un modelo de calidad basado en métricas software del estilo de OpenBRR pero enfocado única y exclusivamente al análisis de las métricas, con una ponderación que permita establecer la calidad y una justificación de la misma, en cuanto a qué métricas son consideradas más importantes y porqué.

\section{Herramientas}

En este apartado se describirán aquellas herramientas que permiten obtener una o varias de las métricas anteriores, y cómo se pueden combinar dichas herramientas para llegar a la evaluación de calidad final. Se contemplarán, igualmente, herramientas de desarrollo propio o adaptaciones a herramientas ya existentes.

\subsection{cccc}

\section{Análisis de datos}

Este apartado permitirá mostrar diversas gráficas y/o reportes comparativos entre dos análisis de calidad realizados sobre distintos proyectos o bien sobre distintas releases de un mismo proyecto.

Así, se podrá mostrar de forma gráfica y cualitativa una comparativa de aquellas métricas más determinantes entre dos sujetos a analizar.


\section{Un ejemplo práctico: Heroes of Allacrost vs. OpenTTD}

En este apartado se realizará el análisis de dos proyectos Open Source cualesquiera, de un tamaño similar en líneas de código, escritos en C++. De esta forma, se popdrán concretar los aspectos vistos anteriormente en una comparativa real.

Un ejemplo posible es la comparativa entre dos juegos Open Source escritos en C++, Heroes of Allacrost y OpenTTD, aunque hay otras opciones. Se puede realizar de igual forma en este apartado más de una comparativa.


\section{Mejoras y posibles trabajos futuros}

Finalmente, dentro de esta sección, se recogerán mejoras y posibles trabajos futuros que se podrían realizar para mejorar o extender lo que se haya realizado en este proyecto.

Se contemplarán, por tanto, opciones de extender el modelo de calidad con más métricas, o bien incluir aspectos que no se han incluido, extender el modelo para más lenguajes de programación, etc.

\pagebreak

\pagebreak

\end{document}
